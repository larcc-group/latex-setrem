% Chapter 3
\chapter{Resultados}\label{chap:Results}

Os experimentos irão nesta seção.

\section{Apresentação da Empresa e Histórico}


\section{Desenvolvimento} \label{sec:dev}


\lstlistingname~\ref{lst:python-code} apresenta um exemplo de código em Python.

\begin{lstlisting}[numbers=left, language=Python, style=mycode, caption={Exemplo de código em Python.}, label={lst:python-code}]
import numpy as np
 
def incmatrix(genl1,genl2):
    m = len(genl1)
    n = len(genl2)
    M = None #to become the incidence matrix
    VT = np.zeros((n*m,1), int)  #dummy variable
 
    #compute the bitwise xor matrix
    M1 = bitxormatrix(genl1)
    M2 = np.triu(bitxormatrix(genl2),1) 
 
    for i in range(m-1):
        for j in range(i+1, m):
            [r,c] = np.where(M2 == M1[i,j])
            for k in range(len(r)):
                VT[(i)*n + r[k]] = 1;
                VT[(i)*n + c[k]] = 1;
                VT[(j)*n + r[k]] = 1;
                VT[(j)*n + c[k]] = 1;
 
                if M is None:
                    M = np.copy(VT)
                else:
                    M = np.concatenate((M, VT), 1)
 
                VT = np.zeros((n*m,1), int)
 
    return M
\end{lstlisting}

\lstlistingname~\ref{lst:cpp-code} apresenta um código em C++.

% this is an example including from a file
\lstinputlisting[numbers=left,language=C++,style=mycode,caption={Exemplo de código em C++.},label={lst:cpp-code}]{code/code-example.cpp}


\section{Experimentos} \label{sec:exp}



