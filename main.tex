\documentclass[diss]{template/setrem}



\usepackage[utf8]{inputenc}
\usepackage[table]{xcolor}
\usepackage{multicol}
\usepackage{array} % for defining a new column type
\usepackage{varwidth} %for the varwidth minipage environment
\usepackage{float}
\usepackage{todonotes}
\usepackage{subfigure}
\usepackage{graphicx,url}
\usepackage{lipsum} %generate fake text

\makeatletter
\g@addto@macro{\UrlBreaks}{\UrlOrds}
\makeatother


\usepackage{setspace}
\usepackage{amssymb}
\usepackage{colortbl}
\usepackage{color}
\usepackage{hyperref}
\usepackage{verbatim}
\usepackage{scrextend}
\usepackage{csvsimple}
\usepackage{glossaries}

\usepackage{listings}
\usepackage{xcolor}
\usepackage{threeparttablex}
\usepackage{lscape}
\usepackage{colortbl}
\usepackage{booktabs}

% Coloca a contagem das figuras sequenciais sem considerar capítulos
\usepackage{chngcntr}
\usepackage{tabularx}
\usepackage{amsmath}
\usepackage{datatool}
\usepackage{seqsplit}
\usepackage[toc,page]{appendix}

% The package option longtable redefines the \cline macro to work around a bug in longtable
\usepackage{longtable}
\usepackage{multirow}
\usepackage{titletoc}

\makeatletter
\def\@cline#1-#2\@nil{%
  \omit
  \@multicnt#1%
  \advance\@multispan\m@ne
  \ifnum\@multicnt=\@ne\@firstofone{&\omit}\fi
  \@multicnt#2%
  \advance\@multicnt-#1%
  \advance\@multispan\@ne
  \leaders\hrule\@height\arrayrulewidth\hfill
  \cr
  \noalign{\nobreak\vskip-\arrayrulewidth}}
\makeatother


\counterwithout{figure}{chapter}
\counterwithout{table}{chapter}

\definecolor{color_keywords}{rgb}{0.0, 0.0, 0.44}
\definecolor{color_mykeyword}{RGB}{10, 100, 112}

\definecolor{myblue}{rgb}{0,0.3,0.9}
\definecolor{mygreen}{rgb}{0,0.6,0}
\definecolor{mygray}{rgb}{0.5,0.5,0.5}
\definecolor{mymauve}{rgb}{0.58,0,0.82}
\definecolor{mygray2}{rgb}{0.2,0.2,0.2}
\definecolor{mygray3}{rgb}{0.4,0.4,0.4}



\lstdefinestyle{mycode}{ %
  %backgroundcolor=\color{yellow!05},   % choose the background color; you must add \usepackage{color} or \usepackage{xcolor}
  basicstyle=\linespread{1}\small,        % the size of the fonts that are used for the code \footnotesize
  %lineskip={-1pt},
  breakatwhitespace=false,         % sets if automatic breaks should only happen at whitespace
  breaklines=true,                 % sets automatic line breaking
  captionpos=t,                    % sets the caption-position to top
  commentstyle=\color{gray!90},    % comment style
%  deletekeywords={...},            % if you want to delete keywords from the given language
  escapeinside={\%*}{*)},          % if you want to add LaTeX within your code
  extendedchars=true,              % lets you use non-ASCII characters; for 8-bits encodings only, does not work with UTF-8
  frame=l,                    % adds a frame around the code (bt,l,r) single
  keepspaces=true,                 % keeps spaces in text, useful for keeping indentation of code (possibly needs columns=flexible)
  keywordstyle=\color{black}\bf,       % keyword style
  language=C++,                     % the language of the code
 % morekeywords={input,output,parallel_activity,stream_producer},            % if you want to add more keywords to the set
  keywordstyle=[2]{\color{blue}\bf},
 % keywords=[3]{serial_in_order, parallel, pipeline, run, add_filter, task_scheduler_init},
 % keywordstyle=[3]{\color{blue!70!green}\bf},
  numbers=left,                    % where to put the line-numbers; possible values are (none, left, right)
  numbersep=5pt,                   % how far the line-numbers are from the code
  numberstyle=\tiny\color{mygray}, % the style that is used for the line-numbers
  rulecolor=\color{black},         % if not set, the frame-color may be changed on line-breaks within not-black text (e.g. comments (green here))
  showspaces=false,                % show spaces everywhere adding particular underscores; it overrides 'showstringspaces'
  showstringspaces=false,          % underline spaces within strings only
  showtabs=false,                  % show tabs within strings adding particular underscores
  stepnumber=1,                    % the step between two line-numbers. If it's 1, each line will be numbered
  stringstyle=\color{red},     % string literal style
  tabsize=1,                       % sets default tabsize to 2 spaces
  title=\lstname                   % show the filename of files included with \lstinputlisting; also try caption instead of title
}

% Negrito no título do listing
\captionsetup[lstlisting]{font={bf},labelfont=bf}

%Ajusta a contagem do listing
\AtBeginDocument{% the counter is defined later
  \counterwithout{lstlisting}{chapter}%
}
\makeatletter
\renewcommand{\l@lstlisting}[2]{%
  \@dottedtocline{1}{0em}{1.5em}{\lstlistingname\ #1}{#2}%
}
\makeatother


\title{your title here}
\author{LastName}{FirstName}
%optional
\author{LastName}{FirstName}
\advisor{Dr.}{LastName}{FirstName}
\advisor{Dr.}{LastName}{FirstName}

% Place where the undergraduate thesis will be made.
\location{Três de Maio}{RS}

% Date of undergraduate thesis development/presentation.
\date{Agosto}{2019}

% Course name - defined in setremdefs.sty.
\course{\ctrc}
%\vspace{-4cm}

% Header image of the cover.

%% Sistemas de Informação
\courseheader{\silogo[1]}

%% Redes de Computadores
%\courseheader{\rclogo[1]}

%% Engenharia de Computação
%\courseheader{\eclogo[1]}

\docname{Trabalho de Conclusão de Curso do Bacharelado em Sistemas de Informação - Faculdade Três de Maio - SETREM}

% Mostra a lista de tabelas e figuras com os prefixos corretos.
\titlecontents{table}
  [0em]
  {}
  {\tablename\enspace\thecontentslabel:\enspace\enspace}
  {}
  {\titlerule*[.5pc]{.}\contentspage}

\titlecontents{figure}
  [0em]
  {}
  {\figurename\enspace\thecontentslabel:\enspace\enspace}
  {}
  {\titlerule*[.5pc]{.}\contentspage}

\begin{document}


\maketitle
\input{apro.tex}

% \begin{agradecimentos}
%   agradeço a todos
% \end{agradecimentos}

% \begin{dedicatoria}
%   dedico a todos
% \end{dedicatoria}

\keyword{Sistemas de Informação}
\keyword{Deep Learning}
\keyword{Agricultura}
\keyword{Revisão Sistemática da Literatura}
\begin{resumo}
  \noindent

  O resumo vai aqui...

\end{resumo}


\keywordenglish{Information Systems}
\keywordenglish{Deep Learning}
\keywordenglish{Agriculture}
\keywordenglish{Systematic Literature Review}
\begin{abstract}
  \noindent The abstract goes here ...

\end{abstract}


\begin{singlespaced}
  \listoffigures
\end{singlespaced}

\begin{singlespaced}
  \listoftables
\end{singlespaced}


% \listofmyequations

\begin{listofabbrv}{OR-AC-GAN} % Put the largest abbreviation.
  \setstretch{1}
  \item[ACM] {Association for Computing Machinery}
  \item[AWS] {Amazon Web Services}
  \item[IBGE] {Brazilian Institute of Geography and Statistics}
  \item[IEEE] {Institute of Electrical and Electronics Engineers}
  \item[IoT] {Internet of Things}
\end{listofabbrv}

% 
\tableofcontents

\chapter*[Introdução]{Introdução}\label{chap:intro}
\addcontentsline{toc}{chapter}{INTRODUÇÃO}

Esse template foi implementado como uma extensão do pacote ABNTEX2. Ele compreende todas as normas exigidas pela Faculdade Três de Maio. 
No Capítulo~\ref{cap_exemplos}, são apresentados todos os exemplos necessários para utilizar o template.

\chapter{Research Plan} \label{chap:ResearchPlan}




\section{Theme} \label{sec::Theme}



\subsection{Theme Delimitation} \label{subsec::ThemeDelimitation}

\lipsum[2-3]

\section{General Objective} \label{sec:objective}


\subsection{Specific Objectives}
\begin{enumerate}
    \item aaaaaaaaaaaaaaa
    \item bbbbbbbbbbbbb
    \item CCCCCCCCCC
    \item DDDDDDDDDDD
    \item FFFFFFFFFFFF
    \item GGGGGGG
\end{enumerate}


\section{Justification}\label{sec:justification}



\section{Problem} \label{sec::Problem}



\section{Hypothesis} \label{sec::Hypothesis}
\begin{enumerate}
    \item A is equal to C
    \item D is bigger than G
\end{enumerate}


\section{Methodology} \label{sec:Methodology}



\subsection{Procedures}



\subsection{Techniques}



\section{Budget} \label{sec:budget}

%table example

\section{Schedule of Activities} \label{sec:schedule_activities_table}

%table example
% Chapter 2

\chapter{Background}\label{chap:background}


\section{Business Field}\label{sec:business}


\section{Fundamentals of Computing for the Studied Area}\label{sec:fundamental}

Equation~\ref{eq:my_equation} is an example of an equation in Latex:

\begin{equation}\label{eq:my_equation}
    h_t = f(W^{(hh)}h_{t-1} + W^{(hx)}x_t).
\end{equation}


Figure\ref{fig:diagram} is an example of including a figure.

\begin{figure}[htb]
    \caption{Simple diagram}
    \centering
    \includegraphics[scale=1.9]{img/diagram.pdf}
    \label{fig:diagram}
    \source{Extracted from \cite{larcc}.}
\end{figure}

\cite{GRIEBLER:IJPP:18}


\cite{MACCOOL:structured_patterns:book:12}


\section{Related Work}\label{sec:rw}


\tablename~\ref{tab:my_table} present an example of a Latex table.

\begin{table}[htb]
    \caption{This is a simple example to build a table.}
    \label{tab:my_table}
    \centering
    \begin{tabular}{|c|c|c|c|}
        \hline
         A & B & N & T\\ 
         \hline
         X & y & W & G\\
         \hline
    \end{tabular}
    
\end{table}
% Chapter 3
\chapter{Results}\label{chap:Results}

The experiments and developement goes here...

\section{History and Presentaion of the Organization}


\section{Developements} \label{sec:dev}


\lstlistingname~\ref{lst:python-code} presents a Python code example.

\begin{lstlisting}[numbers=left, language=Python, style=mycode, caption={Python code example.}, label={lst:python-code}]
import numpy as np
 
def incmatrix(genl1,genl2):
    m = len(genl1)
    n = len(genl2)
    M = None #to become the incidence matrix
    VT = np.zeros((n*m,1), int)  #dummy variable
 
    #compute the bitwise xor matrix
    M1 = bitxormatrix(genl1)
    M2 = np.triu(bitxormatrix(genl2),1) 
 
    for i in range(m-1):
        for j in range(i+1, m):
            [r,c] = np.where(M2 == M1[i,j])
            for k in range(len(r)):
                VT[(i)*n + r[k]] = 1;
                VT[(i)*n + c[k]] = 1;
                VT[(j)*n + r[k]] = 1;
                VT[(j)*n + c[k]] = 1;
 
                if M is None:
                    M = np.copy(VT)
                else:
                    M = np.concatenate((M, VT), 1)
 
                VT = np.zeros((n*m,1), int)
 
    return M
\end{lstlisting}

\lstlistingname~\ref{lst:cpp-code} presents a C++ code example.

% this is an example including from a file
\lstinputlisting[numbers=left,language=C++,style=mycode,caption={C++ code example.},label={lst:cpp-code}]{code/code-example.cpp}


\section{Experiments} \label{sec:exp}




\chapter*{Conclusão} \label{chap:concl}


As conclusões vão aqui ...





%\bibliographystyle{template/abnt}
%% para português
\bibliographystyle{template/abnt-pt}
\bibliography{bib/bibliography}




\end{document}
